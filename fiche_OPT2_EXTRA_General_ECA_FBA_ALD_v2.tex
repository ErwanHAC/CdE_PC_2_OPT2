% ------------------------------------------------------------ %
% Date : ??/??/2023
% ------------------------------------------------------------ %

% ------------------------------------------------------------ %
% -----------------  Auteurs et relecteurs  ------------------ %
% Auteur référent : ??
% Auteur 1 : 
% Auteur 2 : 
% Auteur 3 : 
% Auteur 4 : 
% Relecteur I : 
% Relecteur II.1 : 
% Relecteur II.2 : 
% ------------------------------------------------------------ %

% ------------------------------------------------------------ %
% ----------------------  Fiche ?????  ----------------------- %
% Grand thème : ???
% Thème : ???
% ------------------------------------------------------------ %




% •••••••••••••••••••••••••••••••••••••••••••••••••••••••••••••••••••••••••••• %
% •••••••••••••••••••••••••••••••••••••••••••••••••••••••••••••••••••••••••••• %
\sectionFicheEntrainement{Autres entraînements}
% •••••••••••••••••••••••••••••••••••••••••••••••••••••••••••••••••••••••••••• %
% •••••••••••••••••••••••••••••••••••••••••••••••••••••••••••••••••••••••••••• %






% ********************************************************************* %
%                           ENTRAÎNEMENT                                % 
% ********************************************************************* %
% ================ Métadonnées sur l'entraînement ===================== %

\titreEntrainementFacultatif	{Grandeurs caractéristiques d'un signal}
\hauteurLargeurCadreReponse		{6mm}{2cm}
\dureeResolutionFacultative		{1} % 1, 2, 3 ou 4
\basiqueEtTransversal 			{Y}
\calculALaMain					{N}
\nombreColonnesQuestions		{3} % vide, 1, 2, 3, etc.
\avecPlusieursQuestions			{Y} % Y ou N
\initialisationEntrainement
% ===================================================================== %

La valeur d'un signal lumineux sinusoïdal en un point $x$ à un instant $t$ est donnée par les relations
$$ s(x,t) = S_0 \cos \left( 2\pi \left( \frac{t}{T} - \frac{x}{\lambda} \right)\right)
\quad \text{;} \quad
s(x,t) = S_0 \cos \left( 2\pi \left( \nu t - \sigma x \right) \right)
\quad \text{;} \quad
s(x,t) = S_0 \cos \left( \omega t - k x \right) $$
avec $S_0$ son amplitude, $T$ sa période, $\lambda$ sa longueur d'onde, $\nu$ sa fréquence, $\sigma$ son nombre d'onde, $\omega$ sa pulsation temporelle et $k$ sa pulsation spatiale.

Calculer la valeur des différentes grandeurs caractéristiques des signaux suivants
$$s_1 = S_0 \cos \left( 60 t - 28 x\right)
\quad \text{;} \quad
s_2 = S_0 \cos \left( \frac{t}{21} -\frac{t}{32} - \frac{x}{12} +\frac{x}{7}\right)
\quad \text{;} \quad
s_3 = S_0 \cos \left( \frac{3\pi}{5}t + \frac{t}{23} - \frac{\nu_3}{5} x + \frac{\pi}{2} \right).
$$
% =============================================================================== %
\debutEntrainement
% =============================================================================== %

% ^^^^^^^^^^^^^^^^^^^^^^^^^^^^^^^^^^^^^^ %
%               Question                 %
% ^^^^^^^^^^^^^^^^^^^^^^^^^^^^^^^^^^^^^^ %

\begin{enonce}
	$T_1$  (\SI{}{s})
	\end{enonce}
		
	\reponse{$\SI{0,10}{s}$}
		
	%\begin{corrige}
	%\end{corrige}
		
	% ************************************** %

% ^^^^^^^^^^^^^^^^^^^^^^^^^^^^^^^^^^^^^^ %
%               Question                 %
% ^^^^^^^^^^^^^^^^^^^^^^^^^^^^^^^^^^^^^^ %

\begin{enonce}
$\lambda_1$  (\SI{}{m})
\end{enonce}

\reponse{$\SI{0,22}{m}$}

%\begin{corrige}
%\end{corrige}

% ************************************** %


% ^^^^^^^^^^^^^^^^^^^^^^^^^^^^^^^^^^^^^^ %
%               Question                 %
% ^^^^^^^^^^^^^^^^^^^^^^^^^^^^^^^^^^^^^^ %

\begin{enonce}
	$\nu_2$ (\SI{}{s^{-1}})
	\end{enonce}
		
	\reponse{$\SI{9.7}{s^{-1}}$}
		
	%\begin{corrige}
	%\end{corrige}
		
	% ************************************** %

% ^^^^^^^^^^^^^^^^^^^^^^^^^^^^^^^^^^^^^^ %
%               Question                 %
% ^^^^^^^^^^^^^^^^^^^^^^^^^^^^^^^^^^^^^^ %

\begin{enonce}
$\sigma_2$ (\SI{}{m^{-1}})
\end{enonce}

\reponse{$\SI{2.7}{m^{-1}}$}

%\begin{corrige}
%\end{corrige}

% ************************************** %


% ^^^^^^^^^^^^^^^^^^^^^^^^^^^^^^^^^^^^^^ %
%               Question                 %
% ^^^^^^^^^^^^^^^^^^^^^^^^^^^^^^^^^^^^^^ %

\begin{enonce}
	$\omega_3$ (\SI{}{rad \cdot s^{-1}})
	\end{enonce}
		
	\reponse{$\SI{1.9}{rad \cdot s^{-1}}$}
		
	%\begin{corrige}
	%\end{corrige}
		
	% ************************************** %

% ^^^^^^^^^^^^^^^^^^^^^^^^^^^^^^^^^^^^^^ %
%               Question                 %
% ^^^^^^^^^^^^^^^^^^^^^^^^^^^^^^^^^^^^^^ %

\begin{enonce}
$k_3$ (\SI{}{rad \cdot m^{-1}})
\end{enonce}

\reponse{$\SI{0.061}{rad \cdot m^{-1}}$}

%\begin{corrige}
%\end{corrige}

% ************************************** %


% =============================================================================== %
\finEntrainement
% =============================================================================== %






% ********************************************************************* %
%                           ENTRAÎNEMENT                                % 
% ********************************************************************* %
% ================ Métadonnées sur l'entraînement ===================== %

\titreEntrainementFacultatif	{Brouillage}
\hauteurLargeurCadreReponse		{6mm}{5cm}
\dureeResolutionFacultative		{1} % 1, 2, 3 ou 4
\basiqueEtTransversal 			{N}
\calculALaMain					{N}
\nombreColonnesQuestions		{1} % vide, 1, 2, 3, etc.
\avecPlusieursQuestions			{Y} % Y ou N
\initialisationEntrainement
% ===================================================================== %

% mmmmmmmmmmmmmmmmmmmmmmmmmmmmmmmmmmmmmmmmmmmmmmmmmmmmmmmmm %
% 		  Insertion d'une image en regard d'un texte
% mmmmmmmmmmmmmmmmmmmmmmmmmmmmmmmmmmmmmmmmmmmmmmmmmmmmmmmmm %
% --------------------------------------------------------- %
\pourcentageDeLaPartieAGauche   {0.65}
% --------------------------------------------------------- %
%                                                           %
% -------------------- Partie à gauche -------------------- %
%                                                           %
                                \initialisationPartieGauche % 
%                                                           %
%                                                           %
On éclaire un interféromètre de Michelson réglé en lame d'air avec une source ponctuelle émettant deux signaux lumineux de fréquences $\nu_1$ et $\nu_2$, et de longueurs d'onde $\lambda_1$ et $\lambda_2$. En sortie de l'interféromètre l'éclairement est tel que
$$
I(x) = I_0 \left( 1 +  \cos\left( 2\pi\frac{\nu_1 - \nu_2}{c}x \right) \cos\left(  2\pi\frac{\nu_1 + \nu_2}{c}x \right) \right)
$$
avec $x$ l'écart d'un des deux miroirs de l'interféromètre par rapport à l'autre, comme illustré par l'interférogramme ci-contre.

% --------------------------------------------------------- %
%                                                           %
% -------------------- Partie à droite -------------------- %
%                                                           %
\initialisationPartieDroite %
%                                                           %
%                                                           %
\begin{center}
	\subimport{_images/}{brouillage_ECA_v1.tex}
\end{center}
% --------------------------------------------------------- %
\finalisationDuPartageDePage %					
% --------------------------------------------------------- %

% =============================================================================== %
\debutEntrainement
% =============================================================================== %

% ^^^^^^^^^^^^^^^^^^^^^^^^^^^^^^^^^^^^^^ %
%               Question                 %
% ^^^^^^^^^^^^^^^^^^^^^^^^^^^^^^^^^^^^^^ %

\begin{enonce}
	Exprimer la période spatial $\lambda$ de $\cos\left( 2\pi\frac{\nu_1 - \nu_2}{c}x \right)$.
\end{enonce}
	
	\reponse{$\frac{c}{\nu_1 - \nu_2}$}
	
	\begin{corrige}
		La période spatiale de $\cos\left( 2\pi\frac{\nu_1 - \nu_2}{c}x \right)$ est telle que $
		\lambda = \frac{c}{\nu_1 - \nu_2}$.
	\end{corrige}
	
% ************************************** %

% ^^^^^^^^^^^^^^^^^^^^^^^^^^^^^^^^^^^^^^ %
%               Question                 %
% ^^^^^^^^^^^^^^^^^^^^^^^^^^^^^^^^^^^^^^ %

\begin{enonce}
	À l'aide de l'interférogramme donner la valeur de $\Delta \nu = \nu_1 - \nu_2$.
\end{enonce}
	
	\reponse{\SI{6.00e11}{\hertz}}
	
	\begin{corrige}
		D'après la figure on voit que $\frac{\lambda}{2} = \SI{375}{\micro\meter} - \SI{125}{\micro\meter} = \SI{250}{\micro\meter}$ donc $\lambda = \SI{500}{\micro\meter}$.

		Or, on a vu que $\lambda = \frac{c}{\nu_1 - \nu_2}$, donc 
		$
		\Delta \nu = \frac{c}{\lambda} = \frac{\SI{3e8}{m \cdot s^{-1} }}{ \SI{250e-6}{\meter} } = \SI{6.00e11}{\hertz}
		$.
	\end{corrige}
	
% ************************************** %


% ^^^^^^^^^^^^^^^^^^^^^^^^^^^^^^^^^^^^^^ %
%               Question                 %
% ^^^^^^^^^^^^^^^^^^^^^^^^^^^^^^^^^^^^^^ %

\begin{enonce}
	Exprimer la période spatial $\lambda'$ de $\cos\left( 2\pi\frac{\nu_1 + \nu_2}{c}x \right)$.
\end{enonce}
	
	\reponse{$\frac{c}{\nu_1 + \nu_2}$}
	
	\begin{corrige}
		La période spatiale de $\cos\left( 2\pi\frac{\nu_1 + \nu_2}{c}x \right)$ est telle que $
		\lambda' = \frac{c}{\nu_1 + \nu_2}$.
	\end{corrige}
	
% ************************************** %


% ^^^^^^^^^^^^^^^^^^^^^^^^^^^^^^^^^^^^^^ %
%               Question                 %
% ^^^^^^^^^^^^^^^^^^^^^^^^^^^^^^^^^^^^^^ %

\begin{enonce}
	À l'aide de l'interférogramme donner la valeur de $\nu_0 = \frac{\nu_1 + \nu_2}{2}$.
\end{enonce}
	
	\reponse{\SI{6.00e12}{\hertz}}
	
	\begin{corrige}
		D'après la figure on voit que $8\lambda' = \SI{350}{\micro\meter} - \SI{150}{\micro\meter} = \SI{200}{\micro\meter}$ donc $\lambda' = \SI{25}{\micro\meter}$.

		Or, on a vu que $\lambda' = \frac{c}{\nu_1 + \nu_2}$, donc 
		$
		\nu_0 = \frac{c}{2\lambda'} = \frac{\SI{3e8}{m \cdot s^{-1} }}{ 2\times\SI{25e-6}{\meter} } = \SI{6.00e12}{\hertz}
		$.
	\end{corrige}
	
% ************************************** %


% ^^^^^^^^^^^^^^^^^^^^^^^^^^^^^^^^^^^^^^ %
%               Question                 %
% ^^^^^^^^^^^^^^^^^^^^^^^^^^^^^^^^^^^^^^ %

\begin{enonce}
	Déterminer la valeur de $\lambda_1$ en sommant de $2\nu_0$ et $\Delta \nu$.
\end{enonce}
	
	\reponse{\SI{47.6}{\micro\meter}}
	
	\begin{corrige}
		En sommant $2 \nu_0$ et $\Delta \nu$ il vient que
		$
		2 \nu_0 + \Delta \nu = \nu_1 + \nu_2 + \nu_1 - \nu_2 = 2\nu_1
		$, donc $\nu_1 = \nu_0 + \frac{\Delta\nu}{2}$, soit
		$$
		\lambda_1 = \frac{c}{\nu_1} = \frac{\SI{3e8}{m \cdot s^{-1} }}{ \SI{6.00e12}{\hertz} + \frac{\SI{6.00e11}{\hertz}}{2} } = \SI{47.6}{\micro\meter}.
		$$
	\end{corrige}
	
% ************************************** %


% ^^^^^^^^^^^^^^^^^^^^^^^^^^^^^^^^^^^^^^ %
%               Question                 %
% ^^^^^^^^^^^^^^^^^^^^^^^^^^^^^^^^^^^^^^ %

\begin{enonce}
	Déterminer la valeur de $\lambda_2$ soustrayant $\Delta \nu$ à $2 \nu_0$.
\end{enonce}
	
	\reponse{\SI{52.6}{\micro\meter}}
	
	\begin{corrige}
		En soustrayant $\Delta \nu$ à $2 \nu_0$ et  il vient que
		$
		2 \nu_0 - \Delta \nu = \nu_1 + \nu_2 - \nu_1 + \nu_2 = 2\nu_2
		$, donc $\nu_2 = \nu_0 - \frac{\Delta\nu}{2}$, soit
		$$
		\lambda_2 = \frac{c}{\nu_2} = \frac{\SI{3e8}{m \cdot s^{-1} }}{ \SI{6.00e12}{\hertz} - \frac{\SI{6.00e11}{\hertz}}{2} } = \SI{52.6}{\micro\meter}.
		$$
	\end{corrige}
	
% ************************************** %

% =============================================================================== %
\finEntrainement
% =============================================================================== %






% ********************************************************************* %
%                           ENTRAÎNEMENT                                % 
% ********************************************************************* %
% ================ Métadonnées sur l'entraînement ===================== %
\titreEntrainementFacultatif	{La bonne formule}
\hauteurLargeurCadreReponse		{6mm}{1.5cm}
\dureeResolutionFacultative		{1} % 1, 2, 3 ou 4
\basiqueEtTransversal 			{Y}
\calculALaMain					{N}
\nombreColonnesQuestions		{1} % vide, 1, 2, 3, etc.
\avecPlusieursQuestions			{N} % Y ou N
\initialisationEntrainement
% ===================================================================== %

On considère les figures d'interférence suivantes, pour lesquelles on précise le repère cartésien associé à sa description. 

\begin{center}
	\subimport{_images/}{figure_interf_ALD_v1.tex}
\end{center}


% =============================================================================== %
\debutEntrainement
% =============================================================================== %


% ^^^^^^^^^^^^^^^^^^^^^^^^^^^^^^^^^^^^^^ %
%               Question                 %
% ^^^^^^^^^^^^^^^^^^^^^^^^^^^^^^^^^^^^^^ %
\begin{enonce}
	L'intensité de la figure d'interférence n$^\circ$1 est proportionnelle à la quantité : 
	
	\begin{listeQCM3Colonnes}
	\item $1+ \cos \left(\frac{2\pi a x}{\lambda D} \right)$
	\item $1+ \cos \left(\frac{2\pi a y}{\lambda D} \right)$
	\item $1+ \cos \left(\frac{2\pi a z}{\lambda D} \right)$
	\end{listeQCM3Colonnes}

\end{enonce}

\reponse{\reponseA{}}

\begin{corrige}
	La figure est dans le plan ($zOx$). L'ensemble des points d'éclairement constant correspond à des franges linéaires de direction parallèle à l'axe ($Oz$). Autrement les ensembles de points isophases ne dépendent ni de la coordonnée $y$ (figure plane) ni de la coordonnée $z$ (orientation des franges), donc uniquement de la coordonnée $x$. De manière analogue, on peut dire qu'il n'y a aucune oscillation d'éclairement selon la coordonnée $z$. Réponse (a).
\end{corrige}


\begin{enonce}
	L'intensité de la figure d'interférence n$^\circ$2 est proportionnelle à la quantité : 
	
	\begin{listeQCM3Colonnes}
		\item $1+ \cos \left(\frac{2\pi a x}{\lambda D} \right)$
		\item $1+ \cos \left(\frac{2\pi a y}{\lambda D} \right)$
		\item $1+ \cos \left(\frac{2\pi a z}{\lambda D} \right)$
	\end{listeQCM3Colonnes}

\end{enonce}

\reponse{\reponseB{}}

\begin{corrige}
	La figure est dans le plan ($yOz$). L'ensemble des points d'éclairement constant correspond à des franges linéaires de direction parallèle à l'axe ($Oz$). Autrement les ensembles de points isophases ne dépendent ni de la coordonnée $x$ (figure plane) ni de la coordonnée $z$ (orientation des franges), donc uniquement de la coordonnée $y$. De manière analogue, on peut dire qu'il n'y a aucune oscillation d'éclairement selon la coordonnée $z$. Réponse (b).
\end{corrige}


\begin{enonce}
	L'intensité de la figure d'interférence n$^\circ$3 est proportionnelle à la quantité : 
	
	\begin{listeQCM2Colonnes}
	\item $1+ \cos \big( \frac{4\pi n e}{\lambda } \frac{y_O}{\sqrt{x^2+y_O^2+z^2}} \big)$
	\item $1+ \cos \big( \frac{4\pi n e}{\lambda }\frac{z_O}{\sqrt{x^2+y^2 +z_O^2}} \big)$
	\item $1+ \cos \big( \frac{4\pi n e}{\lambda } \frac{x_O}{\sqrt{x_O^2+y^2 +z^2}} \big)$
	\item $1+ \cos \big( \frac{4\pi n e}{\lambda }\frac{y + z}{\sqrt{x_O^2+y_0^2 +z_O^2}} \big)$
	\end{listeQCM2Colonnes}

\end{enonce}

\reponse{\reponseC{}}

\begin{corrige}
	La figure est dans le plan ($zOy$). L'ensemble des points d'éclairement constant correspond à des franges circulaires. Les ensembles de points d'éclairement constant sont définis pour une valeur constante de distance au centre de la figure $r=\sqrt{y^2+z^2}$ issu du point $O$. Pour aller plus loin, la théorie assure que l'éclairement dépend du cosinus de l'inclinaison des anneaux, soit le rapport de l'adjacent sur l'hypoténuse. Réponse (c).
\end{corrige}

% ************************************** %

% =============================================================================== %
\finEntrainement
% =============================================================================== %







% ********************************************************************* %
%                           ENTRAÎNEMENT                                % 
% ********************************************************************* %
% ================ Métadonnées sur l'entraînement ===================== %

\titreEntrainementFacultatif	{Anneaux du Michelson en lame d'air}
\hauteurLargeurCadreReponse		{6mm}{3cm}
\dureeResolutionFacultative		{2} % 1, 2, 3 ou 4
\basiqueEtTransversal 			{Y}
\calculALaMain					{Y}
\nombreColonnesQuestions		{1} % vide, 1, 2, 3, etc.
\avecPlusieursQuestions			{Y} % Y ou N
\initialisationEntrainement
% ===================================================================== %

Les franges d'interférences d'un interféromètre de Michelson en configuration lame d'air sont des anneaux sombres et brillants (voir figure ci-dessous).

On peut schématiser les rayons lumineux en sortie de l'interféromètre avec le schéma optique ci-dessous où $r$ est le rayon d'un anneau brilant, $\theta$ l'angle d'incidence des rayons lumineux et $f'$ la distance focale de la lentille.


% mmmmmmmmmmmmmmmmmmmmmmmmmmmmmmmmmmmmmmmmmmmmmmmmmmmmmmmmm %
% 		  Insertion d'une image en regard d'un texte
% mmmmmmmmmmmmmmmmmmmmmmmmmmmmmmmmmmmmmmmmmmmmmmmmmmmmmmmmm %
% --------------------------------------------------------- %
\pourcentageDeLaPartieAGauche   {0.5}
% --------------------------------------------------------- %
%                                                           %
% -------------------- Partie à gauche -------------------- %
%                                                           %
                                \initialisationPartieGauche % 
%                                                           %
%                                                           %

\begin{center}
	\subimport{_images/}{6_franges_circulaires_FBA.tex}
\end{center}

% --------------------------------------------------------- %
%                                                           %
% -------------------- Partie à droite -------------------- %
%                                                           %
                                \initialisationPartieDroite %
%                                                           %
%                                                           %
\begin{center}
	\subimport{_images/}{6_anneaux_schema_optique_FBA.tex}
\end{center}
% --------------------------------------------------------- %
                               \finalisationDuPartageDePage %					
% --------------------------------------------------------- %


% =============================================================================== %
\debutEntrainement
% =============================================================================== %


% ^^^^^^^^^^^^^^^^^^^^^^^^^^^^^^^^^^^^^^ %
%               Question                 %
% ^^^^^^^^^^^^^^^^^^^^^^^^^^^^^^^^^^^^^^ %

\begin{enonce}
Exprimer le rayon d'un anneau $r$ en fonction de $\theta$ et de $f'$.
\end{enonce}

\reponse{$f' \tan(\theta)$}

\begin{corrige}
	$\tan(\theta)=\frac{r}{f'}$ donc $r=f' \tan(\theta)$
\end{corrige}

% ************************************** %

% ^^^^^^^^^^^^^^^^^^^^^^^^^^^^^^^^^^^^^^ %
%               Question                 %
% ^^^^^^^^^^^^^^^^^^^^^^^^^^^^^^^^^^^^^^ %

\begin{enonce}
On rappelle que l'ordre d'interférences pour le Michelson en configuration lame d'air est : $p=\frac{2ne \cos(\theta)}{\lambda}$ 
avec $n=\num{1.00}$ l'indice optique de l'air ; $e=\num{5.00}\si{\micro\metre}$ et $\lambda=\num{643}\si{\nano\metre}$.

Calculer l'ordre d'interférence $p_0$ dans le cas où l'angle d'incidence est nulle.
\end{enonce}
	
\reponse{$\num{15.6}$}
	
\begin{corrige}
	$p_0=\frac{2ne \cos(0)}{\lambda}=\frac{2ne}{\lambda}$

	Application numérique : $p_0=\frac{2 \times \num{1.00} \times \num{5.00e-6}}{\num{6.43e-7}}=\num{15.6}$
\end{corrige}
	
% ************************************** %

% ^^^^^^^^^^^^^^^^^^^^^^^^^^^^^^^^^^^^^^ %
%               Question                 %
% ^^^^^^^^^^^^^^^^^^^^^^^^^^^^^^^^^^^^^^ %

\begin{enonce}
Si on fait varier $\theta$ entre $\num{0}$ et $\frac{\pi}{2}$, $\cos(\theta)$ :
\begin{listeQCM3Colonnes}
	\item augmente
	\item diminue
	\item reste constant
\end{listeQCM3Colonnes}
\bigskip
\end{enonce}
			
\reponse{\reponseB{}}
			
%\begin{corrige}
%\end{corrige}
			
% ************************************** %

% ^^^^^^^^^^^^^^^^^^^^^^^^^^^^^^^^^^^^^^ %
%               Question                 %
% ^^^^^^^^^^^^^^^^^^^^^^^^^^^^^^^^^^^^^^ %

\begin{enonce}
On rappelle que l'ordre d'interférences d'une frange brillante est un nombre entier.

Quelle est l'ordre d'interférences $p_1$ du premier anneau brillant visible sur l'écran ?
\begin{listeQCM3Colonnes}
	\item 15
	\item 15.5
	\item 16
\end{listeQCM3Colonnes}
\end{enonce}
		
\reponse{\reponseA{}}
		
\begin{corrige}
	Lorsque $\theta$ augmente, $\cos(\theta)$ diminue donc $p$ diminue aussi. Le premier anneau brillant correspond au premier entier de $p$ plus petit que $p_0=\num{15.6}$. Donc $p_1=\num{15}$.
\end{corrige}
		
% ************************************** %

% ^^^^^^^^^^^^^^^^^^^^^^^^^^^^^^^^^^^^^^ %
%               Question                 %
% ^^^^^^^^^^^^^^^^^^^^^^^^^^^^^^^^^^^^^^ %

\begin{enonce}
En déduire le rayon $r_1$ du premier anneau brillant en $\si{\centi\metre}$ sachant que $f'=\num{50,0}\si{\centi\metre}$.
\end{enonce}
			
\reponse{$\num{13.7}\si{\centi\metre}$}
			
\begin{corrige}
	$r_1=f' \tan(\theta_1)=f' \tan\left(\arccos\left( \frac{p_1 \lambda}{2ne}\right)\right)$

	Application numérique : $r_1=\num{50.0e-2}\tan\left(\arccos\left( \frac{15 \times \num{643e-9}}{2 \times \num{1.00} \times \num{5.00e-6}}\right)\right)=\num{13.7}\si{\centi\metre}$
\end{corrige}
			
% ************************************** %

% ^^^^^^^^^^^^^^^^^^^^^^^^^^^^^^^^^^^^^^ %
%               Question                 %
% ^^^^^^^^^^^^^^^^^^^^^^^^^^^^^^^^^^^^^^ %

\begin{enonce}
	Quelle est l'ordre d'interférences $p_2$ du deuxième anneau brillant visible sur l'écran ?
	\begin{listeQCM4Colonnes}
		\item 14
		\item 15
		\item 16
		\item 17
	\end{listeQCM4Colonnes}
	\bigskip
\end{enonce}
				
\reponse{\reponseA{}}
				
\begin{corrige}
	Pour voir le deuxième anneau brillant, il faut que $\theta$ augmente encore donc que $p$ diminue d'un entier : $p_2=14$.
\end{corrige}
				
% ************************************** %

% ^^^^^^^^^^^^^^^^^^^^^^^^^^^^^^^^^^^^^^ %
%               Question                 %
% ^^^^^^^^^^^^^^^^^^^^^^^^^^^^^^^^^^^^^^ %

\begin{enonce}
En déduire le rayon $r_2$ du deuxième anneau brillant en $\si{\centi\metre}$.
\end{enonce}
				
\reponse{$\num{24.2}\si{\centi\metre}$}
				
\begin{corrige}
	$r_2=f' \tan(\theta_2)=f' \tan\left(\arccos\left( \frac{p_2 \lambda}{2ne}\right)\right)$

	Application numérique : $r_2=\num{50.0e-2}\tan\left(\arccos\left( \frac{14 \times \num{643e-9}}{2 \times \num{1.00} \times \num{5.00e-6}}\right)\right)=\num{24.2}\si{\centi\metre}$
\end{corrige}
				
% ************************************** %

% ^^^^^^^^^^^^^^^^^^^^^^^^^^^^^^^^^^^^^^ %
%               Question                 %
% ^^^^^^^^^^^^^^^^^^^^^^^^^^^^^^^^^^^^^^ %

\begin{enonce}
Calculer le rayon $r_{10}$ du dixième anneau brillant en $\si{\metre}$.
\end{enonce}
					
\reponse{$\num{1.47}\si{\metre}$}
					
\begin{corrige}
	$r_{10}=f' \tan(\theta_{10})=f' \tan\left(\arccos\left( \frac{p_{10} \lambda}{2ne}\right)\right)$

	Application numérique : $r_{10}=\num{50.0e-2}\tan\left(\arccos\left( \frac{ \times \num{643e-9}}{2 \times \num{1.00} \times \num{5.00e-6}}\right)\right)=\num{1.47}\si{\metre}$
\end{corrige}
					
% ************************************** %

% =============================================================================== %
\finEntrainement
% =============================================================================== %






% % ********************************************************************* %
% %                           ENTRAÎNEMENT                                % 
% % ********************************************************************* %
% % ================ Métadonnées sur l'entraînement ===================== %

% \titreEntrainementFacultatif	{Sur l'interféromètre de Michelson - configuration coin d'air}
% \hauteurLargeurCadreReponse		{8mm}{2.0cm}
% \dureeResolutionFacultative		{3} % 1, 2, 3 ou 4
% \basiqueEtTransversal 			{N}
% \calculALaMain					{N}
% \nombreColonnesQuestions		{1} % vide, 1, 2, 3, etc.
% \avecPlusieursQuestions			{Y} % Y ou N
% \initialisationEntrainement
% % ===================================================================== %


% % mmmmmmmmmmmmmmmmmmmmmmmmmmmmmmmmmmmmmmmmmmmmmmmmmmmmmmmmm %
% % 		  Insertion d'une image en regard d'un texte
% % mmmmmmmmmmmmmmmmmmmmmmmmmmmmmmmmmmmmmmmmmmmmmmmmmmmmmmmmm %
% % --------------------------------------------------------- %
% \pourcentageDeLaPartieAGauche   {0.55}
% % --------------------------------------------------------- %
% %                                                           %
% % -------------------- Partie à gauche -------------------- %
% %                                                           %
%                                 \initialisationPartieGauche % 
% %                                                           %
% %                                                           %
% Un interféromètre de Michelson repose sur l'association de deux miroirs plans et d'une lame semi-réfléchissante. En configuration coin d'air, les miroirs plans ne sont pas strictement orthogonaux. Le dispositif équivalent est représenté sur le schéma. 

% \smallskip

% On rappelle que la différence de marche $\delta$ correspond à la quantité : $\delta_{SM} = \int_S^M \; n(s) \; \mathrm{d}s$ avec $n$ l'indice optique au niveau de l'abscisse curviligne $s$ associé au trajet du rayon.

% % --------------------------------------------------------- %
% %                                                           %
% % -------------------- Partie à droite -------------------- %
% %                                                           %
%                                 \initialisationPartieDroite %
% %                                                           %
% %                                                           %


% 	\begin{center}
% 		\includegraphics[width=0.7\textwidth]{_images/Coin_air.png}
% 		\end{center}

		

% % --------------------------------------------------------- %
%                                \finalisationDuPartageDePage %					
% % --------------------------------------------------------- %




% % =============================================================================== %
% \debutEntrainement
% % =============================================================================== %


% % ^^^^^^^^^^^^^^^^^^^^^^^^^^^^^^^^^^^^^^ %
% %               Question                 %
% % ^^^^^^^^^^^^^^^^^^^^^^^^^^^^^^^^^^^^^^ %

% \begin{enonce}
% Exprimer la différence de marche $\delta_{SM}$
% \end{enonce}

% \reponse{$2 x \tan \alpha$}

% %\begin{corrige}
% %\end{corrige}

% % ************************************** %


% % ^^^^^^^^^^^^^^^^^^^^^^^^^^^^^^^^^^^^^^ %
% %               Question                 %
% % ^^^^^^^^^^^^^^^^^^^^^^^^^^^^^^^^^^^^^^ %

% \begin{enonce}
% 	Le rapport de l'éclairement émis par $S$ sur l'éclairement reçu $M$ vaut  : 
	
% 	\begin{listeQCM3Colonnes}
% 	\item $\frac{1}{4}$
% 	\item $\frac{1}{2}$
% 	\item $\frac{3}{4}$
% 	\end{listeQCM3Colonnes}

% \end{enonce}

% \reponse{\reponseB{}}

% \begin{corrige}
% 	Chaque rayon franchit 1 fois la lame semi-réflechissante et subit 1 réflexion sur un miroir donc $I_M=I_0 \left(\frac{1}{2^2} + \frac{1}{2^2}\right)=\frac{1}{2} I_0$. Réponse (b).
% \end{corrige}


% % ************************************** %


% % ^^^^^^^^^^^^^^^^^^^^^^^^^^^^^^^^^^^^^^ %
% %               Question                 %
% % ^^^^^^^^^^^^^^^^^^^^^^^^^^^^^^^^^^^^^^ %

% On rappelle que l'éclairement de la figure d'interférence vérifie la formule de Fresnel : $I\left(M\right)=2I_0 \left(1 + \cos \frac{\delta}{2\pi} \right)$.

% \begin{enonce}
% 	 Quelles formes auront les franges d'interférences ?

% 	\begin{listeQCM3Colonnes}
% 	\item bandes rectilignes
% 	\item carrés épais
% 	\item cercles épais
% 	\end{listeQCM3Colonnes}

% \end{enonce}

% \reponse{\reponseA{}}

% \begin{corrige}
% 	Les franges d'interférences sont isophases, donc telles que $\delta$ soit constant, soit des bandes. On observe des franges rectilignes d'égales épaisseur.
% \end{corrige}

% % =============================================================================== %
% \finEntrainement
% % =============================================================================== %