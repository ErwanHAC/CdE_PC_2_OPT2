\begin{tikzpicture}[decoration={markings, mark= at position 0.6 with {\arrow{latex}}},scale=0.55]
    % axe optique
    \draw[thin,->](0,0)--++(14.5,0)node[above right]{$x$};

    % coordonnées des points
    \coordinate (S) at (0.5,0);
    \coordinate (O1) at (3,0);
    \coordinate (O2) at (9,0);
    \coordinate (FY) at (6,0);
    \coordinate (E) at (14,0);
    \coordinate (S1) at (6,0.6);
    \coordinate (S2) at (6,-0.6);
    \coordinate (M) at (14,1.8);
    \coordinate (H) at (6.5,-0.4);
    
    % Fentes d'Young
    \draw[shift={(FY)},thick] (0,-2)--++(0,1.3);
    \draw[shift={(FY)},thick] (0,-0.5)--++(0,1);
    \draw[shift={(FY)},thick] (0,0.7)--++(0,1.3)node[above]{fentes d'Young};

    
    % Tracé des rayons lumineux
    \draw[color=\AEFcouleurRouge,postaction={decorate}, ultra thick] (S) -- (O1|-S2);
    \draw[color=\AEFcouleurRouge,postaction={decorate}, ultra thick] (O1|-S2) -- (S2);
    \draw[color=\AEFcouleurRouge,postaction={decorate}, ultra thick] (S) -- (O1|-S1);
    \draw[color=\AEFcouleurRouge,postaction={decorate}, ultra thick] (O1|-S1) -- (S1);
    \draw[dashed] (M) -- ($(O2)!-3.2cm!(M)$);
    \draw[color=\AEFcouleurRouge, ultra thick] (S2) -- (H);
    \draw[color=\AEFcouleurRouge,postaction={decorate}, ultra thick] (H) -- (9,0.5);
    \draw[color=\AEFcouleurRouge,postaction={decorate}, ultra thick] (9,0.5) -- (M);
    \draw[color=\AEFcouleurRouge,postaction={decorate}, ultra thick] (S1) -- (9,1.7);
    \draw[color=\AEFcouleurRouge,postaction={decorate}, ultra thick] (9,1.7) -- (M);
    \draw[dashed] (S1) -- ($(H)!-1.3cm!(S1)$);

    % Notation des points
    \draw (S) node{$\bullet$} node[above]{$\ptS$};
    \draw (S1) node[above left]{$\ptS_1$};
    \draw (S2) node[below left]{$\ptS_2$};
    \draw (M) node{$\bullet$} node[right]{$\ptM$};%(x,y,0)$};
    \draw (H) node[above, yshift=5pt]{$\ptH$};
    \draw (H) node{$\bullet$};
    
    % Introduction des angles
    \draw (10.45,0.5) to [bend left] (10.5,0);
    \node at (10.85,0.3) [xshift=5pt]{$\theta_1$};
    \draw (6.95,-1.3) to [bend left] (6,-1.5);
    \node at (6.6,-1.75) [yshift=-5pt]{$\theta_1$};
    
    % Ajout des longueurs caractériqtiques
    \draw[dotted] (0.5,-1.5) -- (S);
    \draw[<->,>=latex,color=\AEFcouleurBleu] (0.5,-1.5)--++(2.5,0) node[midway, above]{$f'_1$};
    \draw[<->,>=latex,color=\AEFcouleurBleu] (9,-1.5)--++(5,0) node[midway, above]{$f'_2$};
    \draw[<->,>=latex,color=\AEFcouleurBleu] (5.5,-0.6)--++(0,1.2) node[midway, left, yshift=3pt]{$a$};

    % Ajout des angles droits
    \draw (6.3,-0.45) -- (6.4,-0.7);
    \draw (6.4,-0.7) -- (6.6,-0.62);
    \draw (6.53,-0.9) -- (6.63,-1.1);
    \draw (6.63,-1.1) -- (6.83,-1.01);

    % Lentilles
    \draw[shift={(O1)},thick,<->,>=latex] (0,-2.25)--++(0,4.5) node[above]{$\mathcal{L}_1$};
    \draw[shift={(O2)},thick,<->,>=latex] (0,-2.25)--++(0,4.5) node[above]{$\mathcal{L}_2$};

    % Ecran
    \draw[shift={(E)},thick,->,>=latex] (0,-2)--++(0,4.5)node[above]{$y$};

\end{tikzpicture}