% ------------------------------------------------------------ %
% Date : 17/06/2023
% ------------------------------------------------------------ %

% ------------------------------------------------------------ %
% -----------------  Auteurs et relecteurs  ------------------ %
% Auteur référent : ??
% Auteur 1 : 
% Auteur 2 : 
% Auteur 3 : 
% Auteur 4 : 
% Relecteur I : 
% Relecteur II.1 : 
% Relecteur II.2 : 
% ------------------------------------------------------------ %

% ------------------------------------------------------------ %
% ----------------------  Fiche ?????  ----------------------- %
% Grand thème : ???
% Thème : ???
% ------------------------------------------------------------ %


% ======================================================= % 
% ============== Gestion de la compilation ============== %
% ======================================================= % 
\ifdefined\mainIsLoaded\else
\RequirePackage{import}
\subimport{../../}{_preambule_CdE_PC}

%Ajout 
\makeatother
\usetikzlibrary{decorations.markings,patterns}

\begin{document}
\fi
% ======================================================= % 
% ======================================================= % 



% ************************************* % 
% ******** Classes concernées  ******** %
% ************************************* % 
% 			    Y = oui			        %
% 		        N = non			        %
% ************************************* % 
\pourPSI{Y}
\pourPC{Y}
\pourMP{Y}
\pourMPI{Y}
\pourPT{Y}
\pourTSI{Y}
\pourTPC{Y}
\pourATS{Y}
% ************************************* % 



% ******************************************************* % 
% *************   Gestion des couleurs     ************** %
% ******************************************************* % 

%  Exemple : définition de la couleur bleue de CBD
%  Merci de préfixer vos couleur

% La commande \couleurNBouCouleur prend deux paramètres :
% #1 est la couleur NB ; #2 est la couleur "en couleur"

\def\TMPUNcouleurBleu{\couleurNBouCouleur{black}{blue}}

% ******************************************************* % 




% ============================================================================ %
%                            FICHE D'ENTRAÎNEMENT                              %
% ============================================================================ %
% ======================= Métadonnées sur la fiche =========================== %
\titreFicheEntrainement		{Interférences à deux ondes}
\grandTheme					{Optique}
\numeroFiche				{XYZ01}
\uniqueID					{WZhaxEInvkq} % une chaîne de caractères (a-z, A-Z) aléatoire de 10 caractères.
\nombreColonnesReponses		{2} % 1, 2, 3, etc.
% ============================================================================ %

%%%%%%%%%%%%%%%%%%%%%%%%%%%%%%%%%%%%%%%%%%%%%%%%%%%%%%%%%%%%%%%%%%%%%%%%%%%%%%%%%%%%%%%%%%%%%%
\debutFicheEntrainement                                                                      %
%%%%%%%%%%%%%%%%%%%%%%%%%%%%%%%%%%%%%%%%%%%%%%%%%%%%%%%%%%%%%%%%%%%%%%%%%%%%%%%%%%%%%%%%%%%%%%




% --------------------------------------------------------------------- %
%                              Prérequis                                %
% --------------------------------------------------------------------- %
% ================== Métadonnées sur le prérequis ===================== %
\avecPrerequis{Y} % Y ou N
% ===================================================================== %
\begin{prerequis}
	% Quelques prérequis, très concis !
	% Respectez le format suivant :  
	Onde monochromatique (pulsation, période, nombre d'onde, longueur d'onde). Déphasage.\\
	$\cos (a+b)=\cos(a)\cos(b)-\sin(a)\sin(b)$\\ 
	$\cos (a-b)=\cos(a)\cos(b)+\sin(a)\sin(b)$\\ 
	$\sin (a+b)=\sin(a)\cos(b)+\cos(a)\sin(b)$\\ 
	$\sin (a-b)=\sin(a)\cos(b)-\cos(a)\sin(b)$\\ 
	
	\constantesUtiles
	\begin{listeConstantes}
		\item nombre d'Avogadro : $\mathcal{N}_A=\SI{6,02e23}{\per\mol}$
	\end{listeConstantes}
\end{prerequis}
% --------------------------------------------------------------------- %






% •••••••••••••••••••••••••••••••••••••••••••••••••••••••••••••••••••••••••••• %
% •••••••••••••••••••••••••••••••••••••••••••••••••••••••••••••••••••••••••••• %
\sectionFicheEntrainement{Superposition de signaux lumineux}
% •••••••••••••••••••••••••••••••••••••••••••••••••••••••••••••••••••••••••••• %
% •••••••••••••••••••••••••••••••••••••••••••••••••••••••••••••••••••••••••••• %







% ********************************************************************* %
%                           ENTRAÎNEMENT                                % 
% ********************************************************************* %
% ================ Métadonnées sur l'entraînement ===================== %

\titreEntrainementFacultatif	{Somme de signaux périodiques}
\hauteurLargeurCadreReponse		{8mm}{2.75cm}
\dureeResolutionFacultative		{1} % 1, 2, 3 ou 4
\basiqueEtTransversal 			{Y}
\calculALaMain					{N}
\nombreColonnesQuestions		{1} % vide, 1, 2, 3, etc.
\avecPlusieursQuestions			{Y} % Y ou N
\initialisationEntrainement
% ===================================================================== %

La lumière peut se modéliser par des vibrations lumineuses dont la représentation est une fonction sinusoïdale. On définit les deux signaux suivants :

\begin{itemize}
	\item $s_1 \left( x,t \right)=S_0 \; \cos \left( \omega t - k x\right)$
	\item $s_2 \left(x,t\right)=S_0 \; \cos \left( \omega t - k x + \varphi\right)$
\end{itemize}

\noindent
avec $\omega$ leur pulsation temporelle, $k$ leur pulsation spatiale (module du vecteur d'onde) et $\varphi$ une phase à l'origine. La superposition $s \left(x,t\right)=s_1 \left(x,t\right) + s_2 \left(x,t\right)$ de ses deux vibrations peut se mettre sous la forme :

\begin{equation*}
	s \left(x,t\right) = S_0 \big[ \; f\left(x,t\right)\; \left( 1+\cos \varphi \right) + g\left(x,t\right) \; \sin \varphi \; \big]
\end{equation*}

% =============================================================================== %
\debutEntrainement
% =============================================================================== %


% ^^^^^^^^^^^^^^^^^^^^^^^^^^^^^^^^^^^^^^ %
%               Question                 %
% ^^^^^^^^^^^^^^^^^^^^^^^^^^^^^^^^^^^^^^ %

\begin{enonce}
Exprimer $f\left(x,t\right)$.
\end{enonce}

\reponse{$\cos\left(\omega t - k x\right)$}

\begin{corrige}
	\begin{align*}
		s\left(x,t\right) & = S_0 \; \cos \left(\omega t - k x\right) + S_0 \; \cos \left(\omega t - k x + \varphi\right) \\
		& = S_0 \; \big( \cos \left(\omega t - k x\right) + \cos \left(\omega t - k x + \varphi\right) \big)\\
		& = S_0 \; \big( \cos \left(\omega t - k x\right) + \cos \left(\omega t - k x\right)\cos \varphi - \sin \left(\omega t - k x\right)\sin \varphi \big)\\
		& = S_0 \; \big( \cos \left(\omega t - k x\right) \; \big(1 +\cos \varphi \big) - \sin \left(\omega t - k x\right)\sin \varphi \big)\\
		& = S_0 \big( \; f\left(x,t\right)\; \big( 1+\cos \varphi \big) + g\left(x,t\right) \; \sin \varphi \; \big)
	 \end{align*}
Par identification, on a $f\left(x,t\right)=\cos \left(\omega t - k x\right)$ et $g\left(x,t\right)=-\sin \left(\omega t - k x\right)$
\end{corrige}

% ************************************** %


% ^^^^^^^^^^^^^^^^^^^^^^^^^^^^^^^^^^^^^^ %
%               Question                 %
% ^^^^^^^^^^^^^^^^^^^^^^^^^^^^^^^^^^^^^^ %

\begin{enonce}
Exprimer $g\left(x,t\right)$.
\end{enonce}

\reponse{$-\sin\left(\omega t - k x\right)$}

\begin{corrige}
Voir corrigé de la question précédente.
\end{corrige}

% ************************************** %


% ^^^^^^^^^^^^^^^^^^^^^^^^^^^^^^^^^^^^^^ %
%               Question                 %
% ^^^^^^^^^^^^^^^^^^^^^^^^^^^^^^^^^^^^^^ %
\vspace*{0.4cm}
\begin{enonce}
	Pour quelle valeur de phase $\varphi$ le signal résultant $s\left(x,t\right)$ s'annule ? 
	
	\begin{listeQCM3Colonnes}
	\item $\varphi=0$
	\item $\varphi=\frac{\pi}{2}$
	\item $\varphi=\pi$
	\end{listeQCM3Colonnes}

\end{enonce}

\reponse{\reponseC{}}

\begin{corrige}
	La fonction $s\left(x,t\right)$ s'annule si et seulement si :
	\begin{equation*}
		\begin{cases}
		& 1+ \cos \varphi = 0 \\
		& \sin \varphi = 0
		\end{cases}
	\end{equation*}
Autrement dit :
\begin{equation*}
	\begin{cases}
		& \varphi = \pi \mod 2\pi \\
		& \varphi = \pi \mod \pi
	\end{cases}
\end{equation*}
	On en déduit que l'unique condition d'annulation est $\varphi = \pi \mod 2\pi$.
\end{corrige}

% ************************************** %

% =============================================================================== %
\finEntrainement
% =============================================================================== %


\newpage

% ********************************************************************* %
%                           ENTRAÎNEMENT                                % 
% ********************************************************************* %
% ================ Métadonnées sur l'entraînement ===================== %
\titreEntrainementFacultatif	{\'Etats isophases}
\hauteurLargeurCadreReponse		{8mm}{1.5cm}
\dureeResolutionFacultative		{1} % 1, 2, 3 ou 4
\basiqueEtTransversal 			{N}
\calculALaMain					{N}
\nombreColonnesQuestions		{1} % vide, 1, 2, 3, etc.
\avecPlusieursQuestions			{Y} % Y ou N
\initialisationEntrainement
% ===================================================================== %


% =============================================================================== %
\debutEntrainement
% =============================================================================== %


% ^^^^^^^^^^^^^^^^^^^^^^^^^^^^^^^^^^^^^^ %
%               Question                 %
% ^^^^^^^^^^^^^^^^^^^^^^^^^^^^^^^^^^^^^^ %

Une source émet deux vibrations lumineuses $s\left(x,t\right)=S_0 \; \cos \left(\omega t - k x\right)$ et $s'\left(x,t\right)=S_0 \; \cos \left(\omega t' - k x'\right)$ de période temporelle $T$ (associée à la pulsation $\omega$) et de longueur d'onde $\lambda$ (associée au nombre d'onde $k$).

% ^^^^^^^^^^^^^^^^^^^^^^^^^^^^^^^^^^^^^^ %
%               Question                 %
% ^^^^^^^^^^^^^^^^^^^^^^^^^^^^^^^^^^^^^^ %

\begin{enonce}
	Exprimer le déphasage $\Delta \varphi$ entre $s$ et $s'$ pour $t=t'=t_0$.
	\end{enonce}
	
	\reponse{$k\left(x'-x\right)$}
	
	\begin{corrige}
	$\Delta \varphi = \omega t_0 - k x - \left( \omega t_0 - k x' \right) = k \left(x-x'\right)$
	\end{corrige}
	
	% ************************************** %

% ^^^^^^^^^^^^^^^^^^^^^^^^^^^^^^^^^^^^^^ %
%               Question                 %
% ^^^^^^^^^^^^^^^^^^^^^^^^^^^^^^^^^^^^^^ %

\begin{enonce}
	Comment s'expriment les écarts de positions $\Delta x_n$ tels que $s$ et $s'$ sont isophases ? 
	\begin{listeQCM2Colonnes}
	\item $\Delta x_n = n \lambda$
	\item $\Delta x_n = \left(n+\frac{1}{2}\right) \lambda$
	\item $\Delta x_n = n \frac{\lambda}{2}$
	\item $\Delta x_n = \left(n+\frac{3}{4}\right) \lambda$
	\end{listeQCM2Colonnes}

\end{enonce}

\reponse{\reponseA{}}

\begin{corrige}
	Le déphasage $\Delta \varphi_n$ entre deux positions successives est constant si $k \left(x_n-x_{n+1}\right)= 0 \mod 2\pi = n 2\pi$. Autrement dit : $\Delta x_n = n\frac{2\pi}{k}= n\frac{2\pi \lambda}{2\pi}= n \lambda$. Pour un instant donné, les positions distantes d'un nombre entier de fois la longueur d'onde de la vibration lumineuse sont en phases : réponse (a).
\end{corrige}

% ^^^^^^^^^^^^^^^^^^^^^^^^^^^^^^^^^^^^^^ %
%               Question                 %
% ^^^^^^^^^^^^^^^^^^^^^^^^^^^^^^^^^^^^^^ %

\begin{enonce}
	Exprimer le déphasage $\Delta \varphi$ entre $s$ et $s'$ pour $x=x'=x_0$.
	\end{enonce}
	
	\reponse{$\omega \left(t'-t \right)$}
	
	\begin{corrige}
		$\Delta \varphi = \omega t - k x_0 - \left( \omega t' - k x_0 \right) = \omega \left(t-t'\right)$
	\end{corrige}
	
	% ************************************** %

% ^^^^^^^^^^^^^^^^^^^^^^^^^^^^^^^^^^^^^^ %
%               Question                 %
% ^^^^^^^^^^^^^^^^^^^^^^^^^^^^^^^^^^^^^^ %

\begin{enonce}
	Comment s'expriment les écarts d'instants $\Delta t_n$ tels que $s$ et $s'$ sont isophases ? 
	\begin{listeQCM2Colonnes}
	\item $\Delta t_n = n T$
	\item $\Delta t_n = \left(n+\frac{1}{2}\right) T$
	\item $\Delta t_n = n \frac{T}{2}$
	\item $\Delta t_n = \left(n+\frac{3}{4}\right) T$
	\end{listeQCM2Colonnes}

\end{enonce}

\reponse{\reponseA{}}

\begin{corrige}
	Le déphasage $\Delta \varphi_n$ entre deux instants successifs est constant si $\omega \left(t_n-t_{n+1}\right)= 0 \mod 2\pi = n 2\pi$. Autrement dit : $\Delta t_n = n\frac{2\pi}{\omega}= n\frac{2\pi T}{2\pi}= n T$. Pour une position donnée, les instants séparés d'un nombre entier de fois la période de la vibration lumineuse sont en phases : réponse (a).
\end{corrige}

% ************************************** %

% =============================================================================== %
\finEntrainement
% =============================================================================== %


% •••••••••••••••••••••••••••••••••••••••••••••••••••••••••••••••••••••••••••• %
% •••••••••••••••••••••••••••••••••••••••••••••••••••••••••••••••••••••••••••• %
\sectionFicheEntrainement{\'Eclairement et figures d'interférences}
% •••••••••••••••••••••••••••••••••••••••••••••••••••••••••••••••••••••••••••• %
% •••••••••••••••••••••••••••••••••••••••••••••••••••••••••••••••••••••••••••• %

% ********************************************************************* %
%                           ENTRAÎNEMENT                                % 
% ********************************************************************* %
% ================ Métadonnées sur l'entraînement ===================== %
\titreEntrainementFacultatif	{La bonne formule}
\hauteurLargeurCadreReponse		{8mm}{1.5cm}
\dureeResolutionFacultative		{1} % 1, 2, 3 ou 4
\basiqueEtTransversal 			{Y}
\calculALaMain					{N}
\nombreColonnesQuestions		{1} % vide, 1, 2, 3, etc.
\avecPlusieursQuestions			{N} % Y ou N
\initialisationEntrainement
% ===================================================================== %

On considère les figures d'interférence suivantes, pour lesquelles on précise le repère cartésien associé à sa description. 

\begin{center}
	\subimport{_images/}{figure_interf_ALD_v1.tex}
\end{center}


% =============================================================================== %
\debutEntrainement
% =============================================================================== %


% ^^^^^^^^^^^^^^^^^^^^^^^^^^^^^^^^^^^^^^ %
%               Question                 %
% ^^^^^^^^^^^^^^^^^^^^^^^^^^^^^^^^^^^^^^ %
\begin{enonce}
	L'intensité de la figure d'interférence n$^\circ$1 est proportionnelle à la quantité : 
	
	\begin{listeQCM3Colonnes}
	\item $1+ \cos \left(\frac{2\pi a x}{\lambda D} \right)$
	\item $1+ \cos \left(\frac{2\pi a y}{\lambda D} \right)$
	\item $1+ \cos \left(\frac{2\pi a z}{\lambda D} \right)$
	\end{listeQCM3Colonnes}

\end{enonce}

\reponse{\reponseA{}}

\begin{corrige}
	La figure est dans le plan ($zOx$). L'ensemble des points d'éclairement constant correspond à des franges linéaires de direction parallèle à l'axe ($Oz$). Autrement les ensembles de points isophases ne dépendent ni de la coordonnée $y$ (figure plane) ni de la coordonnée $z$ (orientation des franges), donc uniquement de la coordonnée $x$. De manière analogue, on peut dire qu'il n'y a aucune oscillation d'éclairement selon la coordonnée $z$. Réponse (a).
\end{corrige}

\newpage


\begin{enonce}
	L'intensité de la figure d'interférence n$^\circ$2 est proportionnelle à la quantité : 
	
	\begin{listeQCM3Colonnes}
		\item $1+ \cos \left(\frac{2\pi a x}{\lambda D} \right)$
		\item $1+ \cos \left(\frac{2\pi a y}{\lambda D} \right)$
		\item $1+ \cos \left(\frac{2\pi a z}{\lambda D} \right)$
	\end{listeQCM3Colonnes}

\end{enonce}

\reponse{\reponseB{}}

\begin{corrige}
	La figure est dans le plan ($yOz$). L'ensemble des points d'éclairement constant correspond à des franges linéaires de direction parallèle à l'axe ($Oz$). Autrement les ensembles de points isophases ne dépendent ni de la coordonnée $x$ (figure plane) ni de la coordonnée $z$ (orientation des franges), donc uniquement de la coordonnée $y$. De manière analogue, on peut dire qu'il n'y a aucune oscillation d'éclairement selon la coordonnée $z$. Réponse (b).
\end{corrige}


\begin{enonce}
	L'intensité de la figure d'interférence n$^\circ$3 est proportionnelle à la quantité : 
	
	\begin{listeQCM2Colonnes}
	\item $1+ \cos \big( \frac{4\pi n e}{\lambda } \frac{y_O}{\sqrt{x^2+y_O^2+z^2}} \big)$
	\item $1+ \cos \big( \frac{4\pi n e}{\lambda }\frac{z_O}{\sqrt{x^2+y^2 +z_O^2}} \big)$
	\item $1+ \cos \big( \frac{4\pi n e}{\lambda } \frac{x_O}{\sqrt{x_O^2+y^2 +z^2}} \big)$
	\item $1+ \cos \big( \frac{4\pi n e}{\lambda }\frac{y + z}{\sqrt{x_O^2+y_0^2 +z_O^2}} \big)$
	\end{listeQCM2Colonnes}

\end{enonce}

\reponse{\reponseC{}}

\begin{corrige}
	La figure est dans le plan ($zOy$). L'ensemble des points d'éclairement constant correspond à des franges circulaires. Les ensembles de points d'éclairement constant sont définis pour une valeur constante de distance au centre de la figure $r=\sqrt{y^2+z^2}$ issu du point $O$. Pour aller plus loin, la théorie assure que l'éclairement dépend du cosinus de l'inclinaison des anneaux, soit le rapport de l'adjacent sur l'hypoténuse. Réponse (c).
\end{corrige}

% ************************************** %

% =============================================================================== %
\finEntrainement
% =============================================================================== %


% •••••••••••••••••••••••••••••••••••••••••••••••••••••••••••••••••••••••••••• %
% •••••••••••••••••••••••••••••••••••••••••••••••••••••••••••••••••••••••••••• %
\sectionFicheEntrainement{Interférométrie}
% •••••••••••••••••••••••••••••••••••••••••••••••••••••••••••••••••••••••••••• %
% •••••••••••••••••••••••••••••••••••••••••••••••••••••••••••••••••••••••••••• %

% ********************************************************************* %
%                           ENTRAÎNEMENT                                % 
% ********************************************************************* %
% ================ Métadonnées sur l'entraînement ===================== %

\titreEntrainementFacultatif	{Sur l'interféromètre de Fabry-Perot}
\hauteurLargeurCadreReponse		{8mm}{2.0cm}
\dureeResolutionFacultative		{2} % 1, 2, 3 ou 4
\basiqueEtTransversal 			{N}
\calculALaMain					{N}
\nombreColonnesQuestions		{1} % vide, 1, 2, 3, etc.
\avecPlusieursQuestions			{Y} % Y ou N
\initialisationEntrainement
% ===================================================================== %


% mmmmmmmmmmmmmmmmmmmmmmmmmmmmmmmmmmmmmmmmmmmmmmmmmmmmmmmmm %
% 		  Insertion d'une image en regard d'un texte
% mmmmmmmmmmmmmmmmmmmmmmmmmmmmmmmmmmmmmmmmmmmmmmmmmmmmmmmmm %
% --------------------------------------------------------- %
\pourcentageDeLaPartieAGauche   {0.55}
% --------------------------------------------------------- %
%                                                           %
% -------------------- Partie à gauche -------------------- %
%                                                           %
                                \initialisationPartieGauche % 
%                                                           %
%                                                           %
Un interféromètre de Fabry-Perot repose sur l'association de deux miroirs semi-réfléchissants placés en parallèle l'un de l'autre. On note $n$ l'indice du milieu optique qui règne entre les deux miroirs. En franchissant l'interféromètre, un rayon lumineux incident sera divisé en deux rayons lumineux émergents : l'un qui n'aura subit aucune réflexion, l'autre deux réflexions (voir schéma ci-contre). En sortie de l'interféromètre, une lentille mince convergente permet de focaliser le faisceau parallèle émergent pour faire interférer les rayons en un point $\ptM$ au niveau de l'écran.

\smallskip

En notant $i$ l'angle de réflexion : $\widehat{\ptA \ptB \ptD}=2i$ et $\widehat{\ptB \ptE \ptH}=i$.

% --------------------------------------------------------- %
%                                                           %
% -------------------- Partie à droite -------------------- %
%                                                           %
                                \initialisationPartieDroite %
%                                                           %
%                                                           %


	\begin{center}
		\subimport{_images/}{Fabry_Perot_ALD_v1.tex}
		\end{center}

		

% --------------------------------------------------------- %
                               \finalisationDuPartageDePage %					
% --------------------------------------------------------- %

% =============================================================================== %
\debutEntrainement
% =============================================================================== %


% ^^^^^^^^^^^^^^^^^^^^^^^^^^^^^^^^^^^^^^ %
%               Question                 %
% ^^^^^^^^^^^^^^^^^^^^^^^^^^^^^^^^^^^^^^ %

\begin{enonce}
Exprimer la différence de marche $\delta_{\ptS\ptM}$
\end{enonce}

\reponse{$2ne \cos i$}

\begin{corrige}
$\delta_{\ptS\ptM}=\left(\ptS\ptM\right)_2 - \left(\ptS\ptM\right)_1 = \left(\ptS\ptB\right)+\left(\ptB\ptD\right)+\left(\ptD\ptE\right)+\left(\ptE\ptF\right) + \left(\ptF\ptM\right) - \big( \left(\ptS\ptB\right) + \left(\ptB\ptH\right) + \left(\ptH\ptC\right) + \left(\ptC\ptM\right) \big)$.\\
Or par construction on sait que $\left(\ptE\ptF\right)=\left(\ptH\ptC\right)$ (projection orthogonale) et $\left(\ptC\ptM\right)=\left(\ptF\ptM\right)$ (lentille mince).\\
De plus on remarque que $\left(\ptB\ptD\right)= \left(\ptD\ptE\right) = n\frac{e}{\cos i}$ ; et que $\left(\ptB\ptH\right) = n\ptB\ptE\sin i = 2n e \tan i \sin i = \frac{2ne \sin^2 i}{\cos i}$.\\
Finalement :
\begin{equation*}
	\delta_{\ptS\ptM}= \left(\ptB\ptD\right)+\left(\ptD\ptE\right) - \left(\ptB\ptH\right) = n\frac{e}{\cos i} + n\frac{e}{\cos i} - \frac{2ne sin^2 i}{\cos i} = 2ne (\frac{1-\sin^2 i}{\cos i} ) = 2ne \cos i
\end{equation*} 
\end{corrige}

% ************************************** %


% ^^^^^^^^^^^^^^^^^^^^^^^^^^^^^^^^^^^^^^ %
%               Question                 %
% ^^^^^^^^^^^^^^^^^^^^^^^^^^^^^^^^^^^^^^ %

\begin{enonce}
	Le rapport de l'éclairement émis par $S$ sur l'éclairement reçu $M$ vaut  : 
	
	\begin{listeQCM3Colonnes}
	\item $\frac{1}{4}$
	\item $\frac{5}{16}$
	\item $\frac{2}{3}$
	\end{listeQCM3Colonnes}

\end{enonce}

\reponse{\reponseB{}}

\begin{corrige}
	La rayon supérieur (1) franchit 2 fois un miroir semi-réfléchissant, donc $I_1=I_0/2^2=I_0/4$. Le rayon inférieur (2) franchit 2 fois un miroir et subit deux réflexions, donc $I_2=I_0/2^4=I_0/16$. Finalement : $I_M=I_0 \left(\frac{1}{4} + \frac{1}{16}\right)=\frac{5}{16} I_0$. Réponse (a).
\end{corrige}


% ************************************** %


% ^^^^^^^^^^^^^^^^^^^^^^^^^^^^^^^^^^^^^^ %
%               Question                 %
% ^^^^^^^^^^^^^^^^^^^^^^^^^^^^^^^^^^^^^^ %

On rappelle que l'éclairement de la figure d'interférence vérifie la formule de Fresnel : $I\left(M\right)=2I_0 \left(1 + \cos \frac{\delta}{2\pi} \right)$,

\begin{enonce}
	 Quelles formes auront les franges d'interférences ?
	
	\begin{listeQCM3Colonnes}
	\item bandes rectilignes
	\item carrés épais
	\item cercles épais
	\end{listeQCM3Colonnes}

\end{enonce}

\reponse{\reponseC{}}

\begin{corrige}
	Les franges d'interférences sont isophases, donc telles que $\delta$ soit constant, soit des cercles. On observe des anneaux d'égales inclinaison.
\end{corrige}

% ************************************** %

% =============================================================================== %
\finEntrainement
% =============================================================================== %

\newpage

% ********************************************************************* %
%                           ENTRAÎNEMENT                                % 
% ********************************************************************* %
% ================ Métadonnées sur l'entraînement ===================== %

\titreEntrainementFacultatif	{Sur l'interféromètre de Michelson - configuration coin d'air}
\hauteurLargeurCadreReponse		{8mm}{2.0cm}
\dureeResolutionFacultative		{3} % 1, 2, 3 ou 4
\basiqueEtTransversal 			{N}
\calculALaMain					{N}
\nombreColonnesQuestions		{1} % vide, 1, 2, 3, etc.
\avecPlusieursQuestions			{Y} % Y ou N
\initialisationEntrainement
% ===================================================================== %


% mmmmmmmmmmmmmmmmmmmmmmmmmmmmmmmmmmmmmmmmmmmmmmmmmmmmmmmmm %
% 		  Insertion d'une image en regard d'un texte
% mmmmmmmmmmmmmmmmmmmmmmmmmmmmmmmmmmmmmmmmmmmmmmmmmmmmmmmmm %
% --------------------------------------------------------- %
\pourcentageDeLaPartieAGauche   {0.55}
% --------------------------------------------------------- %
%                                                           %
% -------------------- Partie à gauche -------------------- %
%                                                           %
                                \initialisationPartieGauche % 
%                                                           %
%                                                           %
Un interféromètre de Michelson repose sur l'association de deux miroirs plans et d'une lame semi-réfléchissante. En configuration coin d'air, les miroirs plans ne sont pas strictement orthogonaux. Le dispositif équivalent est représenté sur le schéma. 

\smallskip

On rappelle que la différence de marche $\delta$ correspond à la quantité : $\delta_{SM} = \int_S^M \; n(s) \; \mathrm{d}s$ avec $n$ l'indice optique au niveau de l'abscisse curviligne $s$ associé au trajet du rayon.

% --------------------------------------------------------- %
%                                                           %
% -------------------- Partie à droite -------------------- %
%                                                           %
                                \initialisationPartieDroite %
%                                                           %
%                                                           %


	\begin{center}
		\includegraphics[width=0.7\textwidth]{_images/Coin_air.png}
		\end{center}

		

% --------------------------------------------------------- %
                               \finalisationDuPartageDePage %					
% --------------------------------------------------------- %




% =============================================================================== %
\debutEntrainement
% =============================================================================== %


% ^^^^^^^^^^^^^^^^^^^^^^^^^^^^^^^^^^^^^^ %
%               Question                 %
% ^^^^^^^^^^^^^^^^^^^^^^^^^^^^^^^^^^^^^^ %

\begin{enonce}
Exprimer la différence de marche $\delta_{SM}$
\end{enonce}

\reponse{$2 x \tan \alpha$}

%\begin{corrige}
%\end{corrige}

% ************************************** %


% ^^^^^^^^^^^^^^^^^^^^^^^^^^^^^^^^^^^^^^ %
%               Question                 %
% ^^^^^^^^^^^^^^^^^^^^^^^^^^^^^^^^^^^^^^ %

\begin{enonce}
	Le rapport de l'éclairement émis par $S$ sur l'éclairement reçu $M$ vaut  : 
	
	\begin{listeQCM3Colonnes}
	\item $\frac{1}{4}$
	\item $\frac{1}{2}$
	\item $\frac{3}{4}$
	\end{listeQCM3Colonnes}

\end{enonce}

\reponse{\reponseB{}}

\begin{corrige}
	Chaque rayon franchit 1 fois la lame semi-réflechissante et subit 1 réflexion sur un miroir donc $I_M=I_0 \left(\frac{1}{2^2} + \frac{1}{2^2}\right)=\frac{1}{2} I_0$. Réponse (b).
\end{corrige}


% ************************************** %


% ^^^^^^^^^^^^^^^^^^^^^^^^^^^^^^^^^^^^^^ %
%               Question                 %
% ^^^^^^^^^^^^^^^^^^^^^^^^^^^^^^^^^^^^^^ %

On rappelle que l'éclairement de la figure d'interférence vérifie la formule de Fresnel : $I\left(M\right)=2I_0 \left(1 + \cos \frac{\delta}{2\pi} \right)$.

\begin{enonce}
	 Quelles formes auront les franges d'interférences ?

	\begin{listeQCM3Colonnes}
	\item bandes rectilignes
	\item carrés épais
	\item cercles épais
	\end{listeQCM3Colonnes}

\end{enonce}

\reponse{\reponseA{}}

\begin{corrige}
	Les franges d'interférences sont isophases, donc telles que $\delta$ soit constant, soit des bandes. On observe des franges rectilignes d'égales épaisseur.
\end{corrige}

% =============================================================================== %
\finEntrainement
% =============================================================================== %

% ---------------------------------------------------------------------------- %
%                       Affichage des réponses mélangées                       %
% ---------------------------------------------------------------------------- %
\afficheReponsesMelangees
% ---------------------------------------------------------------------------- %

%%%%%%%%%%%%%%%%%%%%%%%%%%%%%%%%%%%%%%%%%%%%%%%%%%%%%%%%%%%%%%%%%%%%%%%%%%%%%%%%%%%%%%%%%%%%%%
\finFicheEntrainement                                                                        %
%%%%%%%%%%%%%%%%%%%%%%%%%%%%%%%%%%%%%%%%%%%%%%%%%%%%%%%%%%%%%%%%%%%%%%%%%%%%%%%%%%%%%%%%%%%%%%

% ======================================================= % 
% ============== Gestion de la compilation ============== %
% ======================================================= % 
\ifdefined\mainIsLoaded\else
\printReponsesEtCorriges
\end{document}\fi
% ======================================================= % 
% ======================================================= % 